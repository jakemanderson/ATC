\documentclass[12pt, a4paper]{article}

% Load essential packages
\usepackage[top=1in, bottom=1in, left=1in, right=1in]{geometry}
\usepackage{amsmath, amssymb, amsthm}
\usepackage{graphicx}
\usepackage{booktabs}
\usepackage{natbib}
\usepackage{hyperref}
\usepackage{url}

% Set up bibliography in APA style
\bibliographystyle{apalike}  % APA-like style

% Document information
\title{Do Phone Calls Reduce Jail Time?}
\author{Jake Anderson and Ryan Longmuir}
\date{\today}

\begin{document}
\maketitle


\begin{abstract}
We study the causal impact of reducing phone call costs on pretrial detention durations in county jails. Leveraging recent policy changes in Iowa and Massachusetts that sharply lowered or eliminated the price of jail phone calls, we combine administrative jail roster data with an event study framework and a regression discontinuity in time design. We find that the policies reduced average jail spells by 7 to 10 days, with effects concentrated among individuals detained for low-level offenses. Back-of-the-envelope calculations suggest that the social benefits—including taxpayer savings and reduced liberty costs—substantially exceed the forgone revenue from phone call charges. Our findings highlight how frictions in communication access during incarceration can extend detention and generate downstream criminal justice costs.
\end{abstract}

\newpage
\section{Introduction}


% Frictions in any administrative or bureaucratic environment are unavoidable, but when they are significant, it is important to understand how they affect the people processed by that institution. 

% There are many factors which are important to determining the proper handling of an individual who has committed a crime. Contrition, harm, likelihood of committing another offense, and the law are all important inputs. It is well established that ability to pay is also an important input--whether this should or should not be the case is a big debate in the advocacy space. Measures have been taken to attempt to eliminate disparities caused by differences in bail amounts and willingness to pay. 

% The ability to afford bail, legal counsel, and the time cost to achieve the best legal outcome and minimize time in jail and marks on a criminal record is not the only way that differences in wealth enter the sentencing function. 

% Newly incarcerated or jailed individuals differ in their knowledge of the resources available to them, and also differ in when and if those resources are known to them. Navigating the justice system as a first time offender in particular is perilous and inefficient. 

% The first time being in jail is a harrowing and frightening experience. Most individuals who are booked into county jail are not aware of the process, of their rights, and are experiencing a very unusual set of events that lead them to that jail. 

% Individuals in crisis are less likely to make optimal decisions, even if they are fully aware of the rules of the game and generally play the game well. 

% In other crises, individuals have options to obtain support through outsourcing their needs to external parties--their spouse, family members, lawyer, or other support system. 

% The United States criminal justice system is not unique in its harshness but it is specialized in the level of cruelty it will employ to enforce obedience. Officers have high levels of discretion in whether an individual will be permitted to write down phone numbers from their cell phone, take money from their wallet in the jail-provided locker containing their personal effects, and when they will even be permitted to make the phone call they are entitled to (varying by state), if they are entitled to one at all. 

% This friction and extrajudicial punitive measures may be correlated with the important inputs to sentencing, but they may also be correlated to factors which should not be relevant to the sentencing function--officer disposition, time of day, administrative support, type of offense, and jail level prices of phone calls. 

% Phone calls are the most important resource for individuals who are booked into jail. It is a lifeline which connects them to a support system and most importantly, allows them to connect to their family, lawyer, and job. Connecting with family to receive emotional reassurance and financial support is important as many goods and services in carceral facilities require setup and account balance top ups in order to utilize basic functionality, and if bail is an available option, time in jail pre-trial can be reduced significantly. Connecting with a lawyer directly or through family is the most important step to being released, as most people are unfamiliar with the legal system, how to proceed in various legal circumstances, and there are often grave consequences for taking the wrong steps in legal environments. As for employment, a large fraction of jobs in the US will terminate after a single ``no call no show", and so the ability to notify important parties of an incarcerated individual's unavailable status is important for labor market attachment. 

% How long would it take for someone to notice you are gone? When someone is booked into jail, it's often the case that nobody knows where they are.

% Something as simple as a phone call or notification of being booked into jail is not as simple as it seems. There are many barriers to access, and conditional on access, affordability. In order to obtain access, one needs permission to make the phone call, that the collect call receipt is covered by the carrier (often requiring special settings and default to ``no" in many phone plans), congestion and triage, account creation or telecommunications service sign up, and access to money and personal affects to obtain the number and the money to make the call. 

% Conditional on access, while a phone call can be important for those in jail, it may be the case that they simply do not have the funds--on their person or in general--to make the call. Phone calls from jail are part of an exploitative regime of extraction from incarcerated individuals where predatory pricing and markups transfer large sums

Frictions in any administrative or bureaucratic environment are unavoidable, but when they are significant, it is important to understand how they affect the people processed by that institution. The criminal justice system is no exception. There are many factors that are important in determining the proper handling of an individual who has committed a crime. Contrition, harm, likelihood of committing another offense, and the law are all important inputs. It is also well established that ability to pay is an important input. Whether this should or should not be the case is the subject of ongoing debate in the advocacy space, and measures have been taken to attempt to eliminate disparities caused by differences in bail amounts and willingness to pay. 

The ability to afford bail, legal counsel, and the time cost necessary to achieve the best legal outcome and minimize time in jail or marks on a criminal record is not the only way that differences in wealth enter the sentencing function. Newly incarcerated or jailed individuals differ in their knowledge of the resources available to them, and also differ in when and if those resources are known to them. Navigating the justice system as a first-time offender in particular is perilous and inefficient. The first time being in jail is a harrowing and frightening experience. Most individuals who are booked into county jail are not aware of the process, of their rights, and are experiencing a very unusual set of events that have led them to that jail. In such circumstances, individuals in crisis are less likely to make optimal decisions, even if they are fully aware of the rules of the game and generally play the game well. In other crises, people may outsource decisions and obtain support from external parties—their spouse, family members, lawyer, or other elements of their support system.

The United States criminal justice system is not unique in its harshness, but it is specialized in the level of cruelty it will employ to enforce obedience. Officers exercise high levels of discretion in whether an individual will be permitted to write down phone numbers from their cell phone, take money from their wallet in the jail-provided locker containing their personal effects, and when they will even be permitted to make the phone call they are entitled to (varying by state), if they are entitled to one at all. This friction and the imposition of extrajudicial punitive measures may be correlated with the legitimate inputs to sentencing, but they may also be correlated with factors which should not be relevant to the sentencing function—officer disposition, time of day, administrative support, type of offense, and jail-level prices of phone calls.

Phone calls are the most important resource for individuals who are booked into jail. A phone call is a lifeline that connects them to a support system and, most importantly, allows them to connect with their family, lawyer, and job. Connecting with family provides both emotional reassurance and financial support, as many goods and services in carceral facilities require account setup and balance top-ups in order to utilize basic functionality. If bail is an available option, time in jail pre-trial can be reduced significantly. Connecting with a lawyer directly or through family is often the most important step toward release, since most people are unfamiliar with the legal system, the options available to them, and the potentially grave consequences of taking the wrong steps in a legal environment. As for employment, a large fraction of jobs in the United States will terminate an employee after a single “no call no show.” Thus, the ability to notify important parties of an individual’s status is crucial for labor market attachment. In many cases, when someone is booked into jail, nobody even knows where they are. How long would it take for someone to notice you are gone?

Yet something as simple as a phone call or notification of being booked into jail is not simple in practice. There are many barriers to access, and conditional on access, barriers of affordability. In order to obtain access, an individual must have permission to make the call; the recipient must be able to accept a collect call (which many carriers default to blocking); there may be congestion or triage that limits phone availability; detainees may be required to set up accounts with telecommunications providers; and they may need access to money or personal effects in order to retrieve phone numbers or pay for calls. Conditional on access, affordability remains a separate barrier. Phone calls from jail are part of an exploitative regime of extraction from incarcerated individuals, where predatory pricing and steep markups transfer large sums from detainees and their families to private telecommunications vendors. 


\section{Literature Review}

Most existing research on phone calls in jail has focused on the impact of phone calls and family contact on individuals in prison. See discussion \href{https://www.prisonpolicy.org/blog/2021/12/21/family_contact/}{here}. As early as 1972, the California Department of Corrections issued a report \citep{HoltMiller1972} which found that incarcerated individuals who were not visited were six times more likely to recidivate than those who had at least three visitors. A similar study in Hawaii also found that connection to home community was associated with better outcomes \citep{AdamsFischer1976}. Ohlin's earlier work sought to look at predictive factors of parole success \citep{ohlin_selection_1951}. In Florida, only 42\% of inmates received any visits in their final year of incarceration. They found that reduced recidivism was strongly associated with a visitation close to the release date, and that ``each additional visit received during incarceration lowered the odds of two-year recidivism by 3.8 percent'' \citep{BalesMears2008}. Results discussed \href{https://www.prisonpolicy.org/blog/2018/01/30/knox_report/}{here}.

Turning to the specific effect of phone calls, a 2020 study evaluating the Arizona Prison Visitation Project found that phone calls are important for relationship quakity \citep{haverkate_differential_2020}. When video calls were introduced, providers often demand that in person visits be discontinued in order to recoup the cost of the infrastructure. Concerns about contraband are often raised, but investigations have found that it's the staff, not the incarcerated individuals, who are bringing contraband into the facility. Jails and prisons receive financial incentives for implementing technology which is paid for out of the pockets of the loved ones of incarcerated individuals. Technology can reduce the number of officers who are necessary to monitor the in person visits, and also increase the kick backs the jails and prisons receive from the providers of the technology, as well as increasing utilization (and normalization) of such policies, benefitting the private providers of the technology.

Are these impacts only important at high dosage levels, over long periods of time, and for those who have committed more severe offenses? Or does connection to family and resources have impact in the short term? Connection to community is important for concerns about whether incarcerated individuals are remaining connected to their support networks and also have incentive to ``better themselves'' for the return to their home. Norwegian president when talking about reforming those who have committed crimes says that ``if we treat them like animals, we release animals into society when they finish their sentence''[Citation Needed]. 

Many individuals who find themselves in jail due to rash decisions and a lack of understanding of the consequences. 

Traditional economic approaches to crime, beginning with \citep{Becker1968}, conceptualize criminal behavior as the outcome of rational utility maximization under uncertainty, with individuals weighing expected benefits against the probability and severity of sanctions. Yet this framework struggles to explain impulsive or passion-driven offenses, where deliberation is truncated. Insights from behavioral economics emphasize systematic deviations from rationality: for example, \citep{Laibson1997} shows how hyperbolic discounting leads individuals to overvalue immediate gratification relative to future costs, providing a natural explanation for impulsive offending when emotions run high. A complementary line of research in psychology and neuroscience distinguishes between fast, emotional decision-making and slow, deliberative control; \citep{Kahneman2011}'s dual-system model helps explain why crimes of passion occur when ``System 1'' emotional responses overwhelm ``System 2'' rational evaluation. Jens Ludwig and colleagues demonstrate that targeted interventions teaching adolescents to override automatic, impulsive responses—what psychologists call “System 1” behavior—can sharply reduce violent crime: in a randomized field trial with male youth in high-crime Chicago schools, participation in a cognitive-behavioral therapy program reduced violent-crime arrests by approximately 44 percent during the program year \citep{HellerEtAl2013}. Neurocriminology deepens this account by linking impulsive aggression to impaired executive functioning; \citep{Raine1998} document prefrontal cortex dysfunction among violent offenders, suggesting a biological substrate for failures of self-control in high-emotion contexts. Criminological theory highlights the role of situational and emotional triggers: \citep{Agnew1992}'s General Strain Theory argues that stressors such as betrayal, humiliation, or loss generate negative affect that may culminate in violent retaliation, even absent rational cost-benefit calculation. Together, these perspectives extend Becker's framework by emphasizing the interplay of time-inconsistent preferences, emotional dominance, neurobiological deficits, and situational strain in generating impulsive or passion-driven crime.

Could it be that the same impulsive poor decision making for some individuals will lead them to make rash decisions as they interact with law enforcement and in the early but critical stages of jail admission, arraignment meeetings, and plea bargaining? The first phone call made from jail has the potential to connect them to critical resources which will aid in their decision making, and increase rationality, especially when recently booked individuals are under stress, under the influence of alcohol or other substances, and in an unfamiliar environment. Individuals booked in the evening are staying up past their bedtime--according to research featured in the CSCS study guide [Citation needed], staying up for the 17th hour is equivalent to a Blood Alcohol Concentration of 0.05. It's clear to recenly booked offenders that time is important and that the decisions they make have consequences--it may be more salient than ever as they sit in a cell considering the events which led them to this point. 

Even when individuals remain fundamentally rational, decision-making can differ systematically across stressful contexts. For example, \citet{Mani2013} demonstrate that financial scarcity imposes cognitive load, reducing effective bandwidth and altering choices. Similarly, \citet{Carvalho2016} find that stress induced by Hurricane Sandy increased impatience and steepened discounting among affected households, consistent with rational short-term preference shifts. In a conflict setting, \citet{Callen2014} show that battlefield stress in Afghanistan significantly affected risk-taking and patience, illustrating how acute stress alters economic behavior in high-stakes environments. More broadly, \citet{HaushoferFehr2014} review evidence from neuroeconomics that stress hormones such as cortisol systematically shift time and risk preferences, providing a biological channel through which context influences rational decision-making. Together, this literature highlights that crimes of passion or impulsivity can be interpreted not as irrational departures from Becker’s framework, but as rational choices made under altered cognitive and physiological constraints. 


Police are far from perfect classifiers--some individuals may be innocent and wrongfully imprisoned. Committing a crime is traumatic, and being detained, wrongfully or not, is on average a traumatic experience. But the decisions made by recently booked offenders have large consequences. In fact, the decision to accept a plea bargain is often the single most consequential choice a recently booked offender makes, shaping both immediate sentencing and long-run life outcomes. Early economic models conceptualize plea bargaining as a strategic interaction under uncertainty: \citet{GrossmanKatz1983} show that prosecutors can use sentence offers as a screening mechanism, while \citet{Easterbrook1983} characterizes plea negotiations as a market process that enhances efficiency but highlights the asymmetry of power between defendants and the state. Building on this, legal scholars emphasize how limited information and distorted incentives shape decisions. For example, \citet{Bibas2004} documents how defendants frequently face severe time pressure and lack full information, leading them to overweigh the immediate benefit of release relative to the long-term costs of conviction.

Empirical work confirms that plea decisions dramatically alter sentencing outcomes. \citet{BushwayRedlich2012} show that pleading guilty typically yields substantial sentence reductions compared to going to trial, but also carries risks of coercion and wrongful conviction. Experimental evidence from \citet{DervanEdkins2013} demonstrates that even innocent individuals are often willing to accept plea deals under stress, underscoring the vulnerability of defendants during the booking stage. More recent quantitative studies highlight the interaction between pretrial detention and plea outcomes: \citet{Dobbie2018} find that defendants detained pretrial are significantly more likely to accept plea deals, which in turn worsens long-run economic and criminal outcomes. Similarly, \citet{LesliePope2017} show that racial disparities in bail translate into disparities in pretrial detention and subsequent plea decisions. Taken together, this literature underscores that the plea decision is made under intense stress and asymmetry, and yet it determines not only sentencing but also downstream employment, health, and recidivism trajectories. This is all without mentioning the intense escalation of these consequences once \href{https://www.vera.org/news/how-collateral-consequences-keep-people-trapped-in-the-legal-system}{children are involved} the linked article also discusses employment, immigration status, deportation, housing, public benefits, higher education, civil rights, drivers licenses, .


Phone calls help offenders overcome information frictions. Classic work by \citet{Akerlof1970} shows how asymmetric information can generate inefficient market outcomes, while \citet{Stiglitz2000} emphasizes the broader role of information imperfections in shaping economic performance. In the criminal justice context, high phone call costs act as a friction that limits recently booked individuals' ability to transmit information about identity, employment, housing, or access to resources to officers, and to connect with external actors such as family members and lawyers. Without this information, risk assessments and release decisions are noisier, pushing marginal detainees toward pretrial detention and harsher plea outcomes. 

\section{Setting}

Arrests begin with a 911 call, a police report, police investigation, police observation, traffic stop, or ``stop and frisk''. Suspected offenders are put into a police cruiser and then formally process the individual into the criminal legal system. The person will be brought to the local precinct and complete the booking process, which includes a mug shot, fingerprinting, and paperwork. 

An arrest sets in motion a series of steps that unfold quickly but carry significant consequences. After the initial police encounter—whether a stop, call, or observation—the individual is detained and brought to a precinct for “booking.” This process involves paperwork, mugshots, and fingerprinting. In some cases, particularly for low-level offenses, the person may be released directly from the precinct with a summons or desk appearance ticket. For more serious charges, however, they are transferred to Central Booking, where prosecutors review police records to decide whether to file charges. If no charges are filed, the person must be released; otherwise, they remain in detention.

State law requires that the accused appear in court for arraignment within 24 hours of arrest, though delays are common in practice. During this period, individuals are typically held in crowded cells, often without access to counsel or their families. It is only just before the arraignment that most defendants first meet their public defenders. These initial conversations take place in noisy, public holding cells and are necessarily brief, focused on collecting basic facts: the circumstances of the arrest, names and phone numbers of family members, and evidence of community ties such as employment or housing. Defenders also explain the charges, convey any plea offers, and provide guidance about the upcoming hearing. Because defendants are often hungry, tired, and stressed, these decisions are made under particularly difficult conditions.

At arraignment, charges are formally entered, and the accused person enters a plea. If a guilty plea has been negotiated, the case may be resolved immediately. More often, defendants plead not guilty, and the court then sets release conditions. Judges may release individuals on their own recognizance, impose supervised release or drug testing, or require money bail. If bail is imposed and cannot be paid, the person remains in jail until their case is resolved. Although the law directs judges to consider only flight risk when setting bail, in practice perceptions of dangerousness and other subjective factors often play a role. Because even a single night in jail can lead to job loss, family strain, or coercive plea bargaining, this first 24-48 hour period is among the most consequential stages of the criminal legal process. Communication with the outside world—such as the ability to place a phone call—can therefore be decisive in shaping outcomes, helping to verify employment, secure bail, and strengthen the case for release.

Although criminal procedure varies across jurisdictions, the basic sequence of events following an arrest is broadly similar. In New York, the process is highly structured: after arrest and booking at the precinct, individuals are transferred to Central Booking, where prosecutors decide whether to file charges. By law, defendants must then appear for arraignment within 24 hours, though delays are common. It is typically in holding cells just before arraignment that most defendants first meet their public defenders under stressful, crowded conditions. At arraignment, charges are formally entered, pleas may be taken, and judges set conditions of release, including bail. California procedure follows the same basic outline, but with less rigid timing requirements and a stronger emphasis on citation and release for less serious offenses. Defendants may be taken to jail or released with a notice to appear, and subsequent hearings play a larger role in moving the case forward. Plea agreements can be reached at various stages, not just at arraignment, and bail is determined more flexibly depending on the seriousness of the charge. In both states, however, the early stages of arrest and detention are pivotal: decisions made in the first 24-48 hours strongly shape whether defendants remain incarcerated pretrial, how they engage with plea bargaining, and ultimately what long-term consequences they face.

\subsection*{Summary of Iowa's Criminal Defense Process}

Facing criminal charges in Iowa involves several key stages that shape the outcome of a case. The process begins with an arrest and booking, during which law enforcement collects personal information, fingerprints, and a mugshot. Afterward, the defendant proceeds to an arraignment, where they are formally notified of the charges and enter a plea (guilty, not guilty, or no contest). At this stage, a judge determines bail or considers alternatives to detention, including the possibility of release on recognizance.

Following arraignment, the case enters the pre-trial phase, which includes motions such as requests to suppress evidence or dismiss charges. The defense also gains access to the prosecution's evidence through discovery. Many cases are resolved during plea bargaining, where the defense and prosecution negotiate for reduced charges or sentencing in exchange for a guilty plea.

If no agreement is reached, the case goes to trial, where both sides present evidence and examine witnesses before a judge or jury. If convicted, the defendant enters the sentencing phase, in which the judge weighs the seriousness of the crime, prior history, and mitigating circumstances. Alternatives to incarceration, such as probation or community service, may be considered. Finally, convicted defendants may pursue appeals or post-conviction relief, challenging errors at trial or raising claims such as ineffective assistance of counsel.

Release on recognizance (ROR) refers to the practice of allowing a defendant to be released from custody without paying bail, based solely on their promise to return for future court appearances. Judges typically grant ROR in cases where the defendant is considered low-risk, has strong community ties, and poses little threat to public safety.

\subsection{The Myth of ``One Free Phone Call''}

Under California Penal Code \S 851.5, any person who is arrested and booked has the right to make at least three completed telephone calls within three hours of arrest. These calls are free of charge if placed within the local calling area, or at the arrestee's expense if placed outside that area. The calls may be directed to an attorney (including a public defender if the arrestee cannot afford counsel), to a bail bondsman, or to a relative or other person. Calls made to an attorney may not be monitored, recorded, or overheard by law enforcement. In addition, if the arrested individual is a custodial parent, they are entitled to two additional calls to arrange for the care of their child or children. The law requires that facilities display notices informing arrestees of these rights in English and in other languages commonly spoken in the community, and any officer who willfully deprives an arrestee of these rights commits a misdemeanor.

A common misconception is that an arrested individual only receives one phone call. In fact, California law under Penal Code \S 851.5 guarantees the right to at least three completed telephone calls after booking, and no later than three hours after arrest. These calls may be directed to an attorney, a bail bondsman, or a relative or friend. The calls must be free of charge if within the local calling area, or at the arrestee's expense if long-distance, with the option of collect calling for non-local numbers. Attorney calls are explicitly protected from monitoring or recording. 

Importantly, the statute provides additional protections for custodial parents, who are entitled to two extra calls to arrange care for their children. These rights must be clearly posted in police facilities in English and other locally spoken languages, and any officer who willfully denies them commits a misdemeanor. The blog highlights that these protections are absolute, regardless of immigration status, and that violations can have serious consequences for the admissibility of evidence and law enforcement credibility.

California Penal Code \S 851.5 guarantees that an arrested and booked individual has the right to at least three completed telephone calls within three hours of arrest. As explained by Eisner Gorin LLP (2025), these calls may be to an attorney, a bail bondsman, or a relative or friend, and they must be free of charge if placed within the local calling area, or at the arrestee’s expense if outside the area. Importantly, the law requires that the calls be \textit{completed} to count—merely dialing a number does not satisfy the requirement unless a connection is made with the other party. The calls must be offered as soon as physically possible, and not later than three hours after the arrest. Attorney calls are confidential and may not be monitored, recorded, or overheard. In addition, custodial parents are entitled to two extra calls to arrange for the care of their children. These rights must be clearly posted in English and other locally spoken languages, apply regardless of immigration status, and any officer who willfully denies them commits a misdemeanor.

In Florida, ``For example, you may try to phone your attorney after being arrested and taken to the police station. If you could not get a hold of them and wanted to call back in an hour or so, you would have a right to do so. Or, if you were on your way to pick your child up from school and had to make other arrangements, law enforcement should allow you to make a phone call to do so, as well. Regardless of the reason for the phone call, law enforcement cannot prohibit you from contacting people, within reason, outside of the police station.''

Some jails have access to a phone in the holding area, but it is not always available. 



Officer discretion plays a large role in the availability of phone calls. At nearly every stage of early detention, officer discretion influences whether an individual remains in custody or is released. During the arrest, the officer must decide whether to book or cite and release; a phone call that quickly confirms the arrestee’s identity or transportation can tip this decision toward release. At booking, officers consider whether the individual has family support to post bail; the ability to complete a call allows relatives or a bail bondsman to mobilize funds immediately, reducing the likelihood of unnecessary detention. Pretrial officers and risk assessors often rely on noisy signals of stability, such as verified housing or employment; a timely phone call enables detainees to connect officers with employers, landlords, or family members who can confirm stability, lowering the assessed risk score. Finally, supervisors and shift officers decide whether to hold someone overnight or over the weekend; confirmation of a ride home through a phone call can prevent needless stays. In each of these discretionary moments, phone calls reduce information frictions and logistical barriers, narrowing the gap between those with outside support and those without. Thus, access to phone calls serves as a critical check on discretionary detention decisions and directly shapes outcomes for marginal detainees.



\section{Policy}

In recent years, several states have moved to address the high costs of jail phone calls, which historically created substantial financial burdens for detainees and their families. The state of Iowa provides an important example. On March 16, 2021, the Iowa Department of Corrections reduced the price of a standard 15-minute phone call from roughly \$15 to just \$3. This represented a five-fold reduction in cost, effectively shifting the decision-making margin from one dominated by prohibitive prices toward one where communication became financially feasible for a far larger share of incarcerated individuals. The reform was narrow, however: it targeted the \textit{willingness to pay} channel by reducing the price but left in place other access frictions such as limited phone availability, procedural restrictions, and logistical barriers around account set-up and call monitoring. In short, Iowa's policy lowered the cost of participation in the phone market but did not fundamentally alter the rules of access.

Massachusetts took a more expansive approach two years later. On December 1, 2023, the Massachusetts Department of Correction implemented a policy that made all jail phone calls free of charge. Unlike Iowa's partial reform, Massachusetts eliminated the price barrier altogether, effectively collapsing the \textit{willingness to pay} margin. At the same time, the elimination of fees also influenced the access channel indirectly, as it removed the requirement that detainees or their families establish paid accounts or deposit funds in advance. In practice, this meant that incarcerated individuals could place calls without concern about balance minimums or service interruptions tied to payment systems. By combining cost elimination with administrative simplification, Massachusetts simultaneously addressed both the financial and logistical obstacles to communication.

These state-level changes occurred against the backdrop of a long-standing national debate about the regulation of correctional telecommunications. For decades, jails and prisons contracted with a small set of private providers who charged rates far above market levels, generating an estimated \$1.4 billion annually in call revenue. Families often paid upwards of \$5 per 15-minute call, with some facilities charging as much as \$20, despite commercial vendors offering standard rates closer to one cent per minute. Advocacy organizations such as the Prison Policy Initiative, Worth Rises, and the San Francisco Jail Justice Coalition have highlighted these disparities and campaigned for reforms that center the role of communication in maintaining family ties, improving detainee well-being, and supporting successful reentry.

Together, the Iowa and Massachusetts policies illustrate two distinct models of reform. Iowa's price reduction demonstrates how lowering costs can expand participation by reducing the financial strain on detainees and their families, though it leaves intact other barriers to access. Massachusetts, in contrast, demonstrates the implications of a more sweeping reform that fully eliminates charges, thereby addressing both willingness to pay and access simultaneously. 

\section{Conceptual Framework}

\subsection{Access vs. Willingness to pay}


A central conceptual distinction in the study of jail phone calls is between \textbf{access} and \textbf{willingness to pay}. 
Access refers to whether an incarcerated individual can make a phone call at all: this depends on institutional permission, physical availability of phones, congestion, facility rules, and whether the detainee can retrieve contact information or personal effects. Without access, no amount of resources or desire to connect will translate into a completed call. 
Willingness to pay, by contrast, refers to the decision to use phone services given that access is available but calls are costly. Here, detainees and their families weigh the benefits of communication against the financial burden, often at rates far above market price. 
Policy reforms target these frictions differently: reducing prices alters willingness to pay by lowering the marginal cost of calling, while eliminating fees altogether or mandating free calls can affect both access and willingness simultaneously by removing cost barriers and signaling institutional support for communication. 
Understanding these two dimensions is essential for interpreting policy impacts, as access constraints and cost constraints interact but are not the same.

\subsection{Guilty Pleas}



\section{Data}

The primary data for this study come from the Public Safety Lab’s Jail Data Initiative (JDI), which collects daily roster information from jails across the United States. These rosters contain individual-level records that include name, age, booking and release dates, and charges. For the purposes of this project, the analysis focuses on five county jails: Worcester County in Massachusetts and Polk and Scott Counties in Iowa, with two additional jails serving as potential controls. The sample is restricted to individuals for whom both entry and exit dates can be observed, ensuring accurate measurement of spell lengths.

Across these jails, the analysis covers approximately 21,000 unique individuals. In Iowa, the roster data include detailed covariates such as race, sex, top charge, total bond amount, and known aliases, while in Massachusetts the data are more limited, with only name and age systematically recorded. This heterogeneity in data quality across states reflects differences in reporting practices and underscores the importance of using consistent methods to harmonize variables when possible.

Descriptive statistics reveal that the racial and ethnic composition of the incarcerated population varies across the sites. In Iowa, roughly 64 percent of the sample is White and 25 percent is Black, while in Massachusetts the population is more diverse, with 42 percent White, 29 percent Black, and 26 percent Hispanic. These figures differ from national incarceration patterns, where White individuals comprise about 31 percent of the population and Hispanic individuals about 23 percent. Gender distributions also vary somewhat by site, with women making up between 12 and 24 percent of the observed population. Age distributions are broadly similar across sites, although some variation exists, with younger cohorts more represented in certain counties.

In addition to demographic characteristics, the dataset captures information on top charges at booking, which range from low-level misdemeanors to serious felonies. This variation provides an opportunity to study whether the effects of phone call reforms are heterogeneous by offense type. Aliases recorded in the Iowa data highlight another dimension of inmate identity that may complicate record-linking and risk assessment. For example, a single individual may have multiple recorded nicknames, increasing the difficulty of verification for officers making release decisions.

Taken together, these data provide a unique opportunity to study the impact of communication policies in correctional facilities. By combining pre- and post-policy periods for both Iowa and Massachusetts, the dataset allows for quasi-experimental comparisons that exploit temporal variation in reforms. The richness of the JDI data—particularly in Iowa—also enables subgroup analysis along dimensions such as race, gender, and offense type, thereby illuminating how phone call policies may differentially affect various segments of the incarcerated population.

\section{Model}

To evaluate the impact of phone call reforms on detention outcomes, I employ three complementary models: survival analysis, regression discontinuity in time, and an individual-level event study with two-way fixed effects. Each approach captures a different aspect of how policy changes shape jail spells and subsequent behavior.

The first approach uses survival analysis to estimate the probability of release over time. Specifically, I compute Kaplan–Meier survival functions, which estimate the probability that a detainee remains in custody beyond time $t$:
\[
\hat{S}(t) = \prod_{t_i \leq t} \left(1 - \frac{d_i}{n_i}\right),
\]
where $t_i$ are observed event times (release dates), $d_i$ is the number of releases at time $t_i$, and $n_i$ is the number of detainees at risk immediately before $t_i$. This estimator provides a nonparametric picture of how detention durations differ before and after policy changes.

Second, I implement a regression discontinuity design in time (RDiT), which models outcomes around the date of the policy change. Let $c$ denote the policy cutoff date. The specification is:
\[
Y_{it} = \beta_0 + \beta_1 \cdot \mathbf{1}\{t \geq c\} + \beta_2 (t - c) + \beta_3 \cdot \mathbf{1}\{t \geq c\}(t - c) + \varepsilon_{it},
\]
where $Y_{it}$ is the spell length for individual $i$ booked at time $t$. The coefficient $\beta_1$ captures the immediate policy effect at the cutoff, while $\beta_3$ captures any change in slope after the reform. This model identifies local causal effects under the assumption that no other shocks occur precisely at the cutoff date.

Finally, I estimate a two-way fixed effects event study model that uses individuals as the unit of observation and incorporates leads and lags relative to the reform date. The model is:
\[
y_{ijt} = \sum_{k=-m}^{n} \gamma_k \cdot D_{i,t-k} + \alpha_j + \delta_t + \beta X_{ijt} + \varepsilon_{ijt},
\]
where $y_{ijt}$ denotes days in jail for individual $i$ in facility $j$ at time $t$, $D_{i,t-k}$ are indicators for event time relative to the policy date, $\alpha_j$ are jail fixed effects, $\delta_t$ are time fixed effects, and $X_{ijt}$ are covariates such as demographics and charge type. This specification allows for dynamic treatment effects, controls for unobserved heterogeneity, and provides direct checks for pre-policy parallel trends.

Taken together, these models provide complementary perspectives. The Kaplan-Meier estimator captures broad distributional shifts in detention spells, the regression discontinuity isolates immediate causal effects at the policy threshold, and the event study framework traces dynamic impacts over time. By triangulating across these approaches, I obtain a more robust understanding of how reducing the cost of jail phone calls affects detention outcomes.

\subsection{Are Incarcerated Individuals Price Sensitive?}

An important first step in the analysis is to verify that the policy reforms had the intended effect on inmate phone call behavior. This first stage is crucial: if the reductions in calling costs did not meaningfully increase utilization, then the interpretation of downstream effects on jail time would be less clear.

To examine this, I estimate models of the form
\[
Calls_{jt} = \alpha + \beta \cdot \mathbf{1}\{t \geq c_j\} + \gamma_j + \delta_t + \varepsilon_{jt},
\]
where $Calls_{jt}$ denotes the average number of calls (or total minutes) placed in jail $j$ at time $t$, $c_j$ is the date of the phone call policy reform in that facility, $\gamma_j$ are jail fixed effects, and $\delta_t$ are calendar-time fixed effects. The coefficient $\beta$ captures the average change in call volume attributable to the reform. In specifications that pool across facilities, I also estimate a stacked event study that follows Lafortune, Rothstein, and Schanzenbach (2018), aligning treatment dates across jails to produce a common set of event time indicators.

The results show sharp increases in phone call usage exactly at the time of reform. In Iowa, call volumes rose significantly after the reduction from \$15 to \$3 per 15 minutes, suggesting strong price sensitivity on the willingness-to-pay margin. In Massachusetts, where calls were made free, the effect was even larger: call volumes increased both in frequency and in duration, consistent with the removal of cost barriers and account restrictions. These patterns appear clearly in the event study plots, which show no evidence of differential pre-trends prior to the reforms.

Taken together, the first stage confirms that the policies successfully shifted calling behavior, validating the key assumption that downstream effects on detention length and recidivism can be interpreted as arising from improved access to communication rather than unrelated contemporaneous factors.

Evidence from multiple states indicates that the effect of reducing or eliminating the cost of jail phone calls is highly consistent. Connecticut, California, Massachusetts, and other jurisdictions that implemented reforms all saw sharp increases in phone call utilization immediately after the policies took effect. These jumps in both call frequency and total minutes suggest that the demand for communication among incarcerated individuals is strongly constrained by price, and that lowering or eliminating charges reliably translates into greater use of phone services across diverse institutional contexts.



\section{Results}

The results provide consistent evidence that reductions in phone call costs decrease the length of jail spells. In Iowa (Polk County), the regression discontinuity estimates indicate an immediate drop of approximately 7.4 days in average detention length following the March 2021 policy change. In Massachusetts (Worcester County), where the policy eliminated all charges for calls in December 2023, the immediate decline was larger, at about 10.6 days. These effects are statistically significant and robust across alternative model specifications. The post-policy slope terms suggest that the reductions in jail time persisted beyond the immediate cutoff, with detention lengths continuing to decline at a faster rate in the months after the reform.

The Kaplan-Meier survival analysis reinforces these findings by showing a visible leftward shift in the survival curve after the reforms. In both Iowa and Massachusetts, detainees were released more quickly, with the largest gains concentrated in the early weeks of custody. The distributional evidence suggests that the benefits of cheaper or free phone calls accrue most strongly to those on the margin of release, where communication with family members, employers, or bail providers can directly influence decisions by officers, judges, or supervisors.

The event study estimates provide additional support by documenting dynamic treatment effects over time. Placebo tests using pre-reform leads show no evidence of differential trends prior to the policy changes, supporting the validity of the design. In the months following reform, the treatment effects grow stronger, consistent with both an immediate reduction in detention and continued improvements as families and detainees adjusted to the new calling environment. Subgroup analyses suggest that the largest effects are observed for individuals booked on low-level charges and those with relatively small bond amounts, groups for whom timely communication with family or a bail bondsman can make the difference between immediate release and extended detention.

Taken together, the results indicate that policies reducing the cost of jail phone calls meaningfully shorten detention spells. Iowa's partial reform demonstrates that lowering the price from very high levels to a more reasonable rate produces measurable gains, while Massachusetts' elimination of fees produces even larger impacts by removing both willingness-to-pay and access barriers. These findings highlight the importance of communication as a mechanism in pretrial detention decisions, and suggest that reforms targeted at reducing information frictions can substantially alter carceral outcomes.

\section{Next Steps}

So far, this project has yielded interesting and statistically significant results for the Regression Discontinuity in Time (RDiT) design for several jails of interest in Iowa and Massachusetts. The next phase of the project involves aggregating across many different jails to form a stacked event study design as in \citep{LafortuneRothsteinSchanzenbach2018}.
\begin{enumerate}
    \item Scrape jails with sufficient coverage to have stable controls before and after policy
    \item Obtain via Freedom of Information Act (FOIA) requests better data
    \item Run the stacked event study design with the different policy changes
    \item Obtain court records about guilty pleas (even in aggregate)
    \item Find out if I can get any more granular data about phone call utilization
\end{enumerate}







\newpage
\bibliography{/Users/jakeanderson/Documents/research/ATC/resources/references}

\end{document}